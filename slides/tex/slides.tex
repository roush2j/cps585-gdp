\documentclass[aspectratio=169]{beamer}
\usetheme{metropolis}           % Use metropolis theme

\usepackage{amsmath}
\usepackage{xcolor,colortbl}
\newcommand{\vect}[1]{\boldsymbol{#1}}
\newcommand{\expc}[1]{\left< #1 \right>}

\title{Predicting Gross Domestic Product}
\subtitle{Using Vector Autoregression}
\date{\today}
\author{Keith Siopes, John Roush, and Neeraj Rajesh}
\institute{Central Michigan University}
\begin{document}
    
    \maketitle

    \section{Real GDP and Economic Indicators}
    \begin{frame}{GDP}
    \emph{Gross Domestic Product} measures the value of all goods and
    services produced in a country in a fixed period.    
    \[GDP = consumption + investment + (exports - imports)\]        
    Consumption and investment are further divided between private
    interests and government activities.
    
    \emph{Real GDP} corrects for inflation (measured independently) to make
    GDP comparable from year to year.        
\end{frame}

\begin{frame}{Indicators}
    \begin{description}
        \item [Treasury Interest Rate]
            Specifically, for 3-month treasury bills.  Set by the treasury to
            influence economic activity.
        \item [Total Capacity Utilization]
            The value of industrial goods produced compared to the available
            industrial capacity.
        \item [industrial productivity]
        \item [Purchasing Manager Index]
            An aggregate indicator of health in the manufacturing sector. 
            Based on new orders, inventory levels, production, supplier 
            deliveries and the employment environment.
        \item [real retail sales]
    \end{description}
\end{frame}

\begin{frame}{Indicators}
    \begin{description}
        \item [Labor Force Participation]
            The fraction of the eligible civilian labor force that is 
            employed or seeking employment.
        \item [Real Personal Income]
            Average individual income, adjusted for inflation.
        \item [Non-farm employment]
        \item [Outstanding Mortgage Debt]
        \item [Consumer Debt]
    \end{description}
\end{frame}

\begin{frame}{Indicators}
    \begin{description}
        \item [Private Consumption Expenditure]
        \item [Government Consumption Expenditure]
        \item [Private Investment]
    \end{description}
\end{frame}

\begin{frame}{Data Sources}
    Data drawn from the Federal Reserve Bank of St. Louis.
    https://research.stlouisfed.org/fred2
    
    All data is quarterly (3-month) average and already seasonally adjusted.
\end{frame}

    \section{Autoregression}
    \begin{frame}{Scalar Autoregression}
    A time-series model where each value depends linearly on the $p$ 
    previous values:
    \begin{align*} x_t
        &= \beta + \sum_{i=1}^p \phi_i x_{t-i} + \epsilon_t \\
        &= \beta + \phi_1 x_{t-1} + \phi_2 x_{t-2} + \ldots + \phi_p x_{t-p} + \epsilon_t
    \end{align*}
    where $\epsilon_t$ is a noise term with zero mean, constant variance,
    and zero auto-covariance:
    \begin{align*}
        \expc{\epsilon_t} &= 0 \\
        \expc{\epsilon_t^2} &= \sigma_\epsilon^2 \\
        \expc{\epsilon_t \cdot \epsilon_{t-T}} &= 0 \text{ for } T \neq 0
    \end{align*}
     
     AR is a useful model if $x_t$ is \emph{stationary}: statistical 
     properties like mean and variance are independent of $t$.
\end{frame}

\begin{frame}{Vector Autoregression}
    An $n$-variable model in which each variable depends linearly on the
    $p$ past values of all variables:
    \[ \vect{x}_t
        = \vect{\beta} + \sum_{i=1}^{p} \vect{\phi}_{i} \cdot \vect{x}_{t-i} + \vect{\epsilon}_t
     \]
     where $\vect{x}_t$, $\vect{\beta}$, and $\vect{\epsilon}_t$ are now 
     vectors of $n$ variables and $\vect{\phi}$ is now an $n \times n$ 
     matrix of coefficients.
     %NOTE and \epsilon must have a positive-definite covariance
    
    For example, a VAR(2) model of 2 variables $a$ and $b$:
    \[ 
        \begin{pmatrix} a_t \\ b_t \end{pmatrix} = 
        \begin{pmatrix} \beta_{a} \\ \beta_{b} \end{pmatrix} +
        \begin{pmatrix}
            \phi_{1aa} & \phi_{1ab} \\
            \phi_{1ba} & \phi_{1bb}
        \end{pmatrix} \cdot 
        \begin{pmatrix} a_{t-1} \\ b_{t-1} \end{pmatrix} + 
        \begin{pmatrix}
            \phi_{2aa} & \phi_{2ab} \\
            \phi_{2ba} & \phi_{2bb}
        \end{pmatrix} \cdot 
        \begin{pmatrix} a_{t-2} \\ b_{t-2} \end{pmatrix} +
        \begin{pmatrix} \epsilon_{a,t} \\ \epsilon_{b,t} \end{pmatrix}
    \]
\end{frame}

\begin{frame}{Stationarity}
    Autoregressive models only apply to \emph{stationary} time series:
    the mean, variance, and autocovariance must be constant over time.
        \begin{align*}
            \expc{x_t} &= \mu \\
            \expc{(x_t - \mu)^2} &= \sigma_x^2 \\
            \expc{(x_t - \mu) \cdot (x_{t-T} - \mu)} &= C(T)
        \end{align*}             
    This condition is true only if the roots of the characteristic 
    polynomial lie outside the unit circle:        
    \[ 1 - \phi_1 z - \phi_2 z^2 - \ldots - \phi_p z^p = 0 \]
    \[ \| z \| > 1 \]
    
    We can use the \textbf{Augmented Dickey-Fuller} test to check this 
    assumption for a given model.
\end{frame}

\begin{frame}{Choosing the Order}
    Akaike Information Criterion to choose $p$   
\end{frame}
    
    \section{Models}
    \begin{frame}{Data Transformations}
    Real GDP and all of the indicators have strong exponential trends.
    
    To obtain stationary data for the VAR model, we first take the log
    of each variable, and then the first difference:
    %TODO
    
\end{frame}

\begin{frame}{1}
    PCE, GCE, PI
\end{frame}
    
    \section{Conclusions}    
        \begin{frame}{Conclusions}
        \begin{itemize}
        \item The model appears to have predicted past values accurately.
        \item Economic growth is expected to continue at least for the next 4 quarters.
        \item The 95\% confidence interval grows quite quickly and makes long-term predictions meaningless.
        \item The outliers are of particular importance; recessionary periods
            are the most useful economic event to forecast.
        \end{itemize}
    \end{frame}
    
    \begin{frame}{Possible further work}
        \begin{itemize}
        \item This model is very basic - there are many possible combinations
            of variables to try with VAR.
        \item Cross-validation (70\% training, 30\% testing)
        \item Other model structures like Vector Error Correction or 
            Vector Autoregressive Moving Average.
        \end{itemize}
    \end{frame}
    
    \begin{frame}
        \begin{center} \huge \bf Questions? \end{center}
    \end{frame}

\end{document}
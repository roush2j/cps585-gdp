\section{GDP and Other Economic Indicators}

    \define{Gross Domestic Product} measures the value of all goods and
    services produced in a country during a fixed period.  GDP is traditionally defined
    as the sum of consumption expenditures, investments, and the trade balance (the
    difference between export and import) \cite{samuelson}:
    \[ GDP = consumption + investment + (exports - imports) \]
    Consumption and investment are further divided between private
    interests and government activities.

    \define{Real GDP} corrects for inflation (measured independently) to make
    GDP comparable from year to year.  We will naturally be forecasting real GDP,
    and unless otherwise specified ``GDP'' in this paper can be assumed to 
    mean ``real GDP''.

    There is a lot of complexity hidden in succinct definition above.
    Consumption, Investment, and Trade are umbrella terms covering many 
    variables.  Private consumption, for example, includes Retail Sales and 
    Private Consumption Index, each an important economic indicator in its own 
    right.  We shall use some of these other variables in our model,
    so without going into too much detail we name a few and sketch their definitions
    in section \ref{sec:indicators} below.

\subsection{Other Economic Indicators} \label{sec:indicators}

    \define{Real Private Consumption Expenditures} is the main component of 
    Private Consumption.  A key component of rPCE is the \define{Private 
    Consumption Index}: the total amount of money spent on goods and 
    services by private consumers. Also important is \define{Gross Retail 
    Sales}: the amount of money spent on \emph{retail} goods and services by 
    private consumers stores, notable as an indicator of faith in the currency. 

    \define{Gross Private Domestic Investment} is the main component of Private 
    Investment.  It is a particularly important component of GDP because it 
    provides an indicator of future capital availability in the economy. GPDI 
    includes 3 types of investment:
    \begin{itemize}
        \item \define{Capital Investment} is expenditures by firms on capital 
        such as tools, machinery, and factories. 
        \item \define{Residential Investment} is expenditures on residential 
        structures and residential equipment that is owned by landlords and 
        rented to tenants. 
        \item \define{Inventory Investment} is change of company inventories 
        during the survey period.
    \end{itemize}
        
    \define{Real Government Consumption Expenditure} is a major metric of 
    government spending.  It covers government purchases of goods and services 
    which is not a transfer payment of money collected in taxation from one 
    group in society to another (such as welfare). Much day-to-day health and 
    education expenditure counts as government consumption. 

    There are many more major and minor metrics that could be considered:
    \begin{itemize}
        \item \define{Outstanding Mortgage Debt} is the amount of mortgage debt held by the
        private sector. 
        \item \define{Consumer Debt} is short term debt used to fund consumption and 
        includes debts incurred on purchase of goods that are consumable and/or do 
        not appreciate. 
        \item \define{Civilian Labor Force Participation}
        is the fraction of the eligible civilian labor force (non-military, 
        non-institutionalized adults) that is employed or seeking employment.
        \item \define{Non-farm Employment} employment by goods, construction 
        and manufacturing companies in the US. It doesn't include private household 
        employees or nonprofit organization employees. 
        \item \define{Production Manager Index} is based on surveys to businesses of five 
        variables: new orders, inventory levels, production, supplier deliveries 
        and the employment environment. 
        \item \define{Real Personal Income} is the average income of private individuals 
        after adjusting for inflation. 
        \item \define{Industrial Production Index} measures the output from the 
        manufacturing, mining, electric and gas industries.
    \end{itemize}
    
    While we tested models built on several combinations of these variables, in this
    paper we will focus solely on a model using rPCE, rGCE, and GPDI.
\begin{abstract}
    We discuss various economic indicators and the theory of predicting them
    using autoregressive models.  We then construct a VAR(4) model on a 
    small selection of indicators to predict Gross Domestic Product.  
    It matches historical GDP data well and predicts consistent future growth,
    but we conclude that it is too simple to have compelling predictive power.
\end{abstract}

\section{Introduction}
    Nobel laureate Paul A. Samuelson and economist William Nordhaus said that
    \begin{quote} 
        GDP and the rest of the national income accounts may seem to be arcane 
        concepts, [but] they are truly among the great inventions of the 
        twentieth century \cite{samuelson}.
    \end{quote}
    
    \define{Gross Domestic Product} is an overall estimate of the size of an 
    economy, in terms of total productive output. As such, it is of great 
    importance to policy makers and central banks to know its trend and foresee 
    possible changes.  This facilitates accurate assessment of the future state 
    of the economy - whether it be heading toward expansion or contraction - 
    such that preemptive actions can be taken as appropriate. Reliable 
    forecasting of GDP is a very important tool in the macro-economics toolbox.

    Unfortunately, national economies are profoundly complex and it is not 
    trivial to model them accurately, let alone predict their behavior. Like 
    weather forecasts, GDP forecasts are inherently uncertain and 
    inevitably short-term.

    We look briefly at some previous work on the subject of predicting GDP and 
    then attempt to do the same.  We use a VAR(4) model on a small selection of 
    economic indicators.  The results are consistent but not particularly 
    insightful and we conclude that more sophisticated models are needed.